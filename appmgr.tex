\section{appmgr}

appmgr的启动

\begin{verbatim}

static const char* argv_appmgr[] = { "/system/bin/appmgr" };
appmgr_hnds[0] = appmgr_req_srv;
devmgr_launch(fuchsia_job_handle, "appmgr", countof(argv_appmgr),
                argv_appmgr, NULL, -1, appmgr_hnds, appmgr_ids,
                appmgr_hnd_count, NULL, FS_FOR_APPMGR);

    env是空
    launchpad_create(job_copy, name, &lp);
        launchpad_create_with_jobs(job, xjob, name, result);
            zx_process_create(creation_job, name, name_len, 0, &proc, &vmar);
            launchpad_create_with_process(proc, vmar, &lp)
    
    file_vmo = devmgr_load_file("/system/bin/appmgr")
        bootfs_open(&bootfs, path + 6, &vmo);
            zx_vmo_clone(bfs->vmo, ZX_VMO_CLONE_COPY_ON_WRITE,
                          e->data_off, e->data_len, &vmo)
    launchpad_load_from_vmo(lp, file_vmo)
        launchpad_file_load_with_vdso(lp, vmo);
            launchpad_file_load(lp, vmo);
                launchpad_elf_load_body(lp, first_line, to_read, vmo);
                    不是脚本,作为elf 加载
                    elf_load_start(vmo, hdr_buf, buf_sz, &elf)
                    elf_load_get_interp(elf, vmo, &interp, &interp_len)
                    handle_interp(lp, vmo, interp, interp_len)
                        加载动态连接器
                        elf_load_start(interp_vmo, NULL, 0, &elf)
                        elf_load_finish(lp_vmar(lp), elf, interp_vmo,
                                 &segments_vmar, &lp->base, &lp->entry);
                        真正的exe vmo会作为HND_EXEC_VMO类型的handle传给ld.so


            launchpad_load_vdso(lp, ZX_HANDLE_INVALID);
                launchpad_elf_load_extra(lp, vmo, &lp->vdso_base, NULL);
                    elf_load_start(vmo, NULL, 0, &elf)
                    elf_load_finish(lp_vmar(lp), elf, vmo, NULL, base, entry)

            launchpad_add_vdso_vmo(lp);

    launchpad_go(lp, proc, &errmsg)
        launchpad_start(lp, &h);
            prepare_start(lp, "initial-thread", to_child, &thread, &sp);
                zx_thread_create(lp_proc(lp), thread_name,strlen(thread_name), 0, thread);
                send_loader_message(lp, *thread, to_child);
                    把arg, env, names作为消息发送给新进程
                    names包含了那些appmgr可见的namespace的路径,定义在devmgr-fdio.c:FSTAB
                
                create stack vmo for the new thread

            zx_process_start(proc, thread, lp->entry, sp, child_bootstrap, lp->vdso_base);

动态链接库是libc.so. 入口在dl-entry.S文件里。这里会链接真正的可执行文件,找到
它的入口地址,进入它。实现在zircon/third_party/ulib/musl/ldso/dynlink.c

这里推测,fuchsia认为动态链接是libc的功能,不是libzircon的功能。

可执行文件的入口在zircon/third_party/ulib/musl/arch/aarch64/Scrt1.S

__libc_start_main()设置一些全局变量
__environ
__zircon_process_self;
__zircon_vmar_root_self;
__zircon_job_default;
main_thread_handle

__allocate_thread分配线程需要的内存

进入start_main(),调用构造函数。进入main!

argv = "/system/bin/appmgr"

async::Loop loop(&kAsyncLoopConfigMakeDefault)
    async_loop_create(config, &loop_)
    
fs::SynchronousVfs vfs(loop.async())
    缺省构造函数

component::RootLoader root_loader
    一个服务,用的fidl接口, 重载LoadComponent()

directory->AddEntry(
    component::Loader::Name_, //"component.Loader"
    fbl::AdoptRef(new fs::Service([&root_loader](zx::channel channel) {
        root_loader.AddBinding(
            fidl::InterfaceRequest<component::Loader>(std::move(channel)));
        return ZX_OK;
    })));
    让"component.Loader"对应到一个fs::Service上。Service是Vnode的一种。
    Connector = fbl::Function<zx_status_t(zx::channel channel)>;


vfs.ServeDirectory(directory, std::move(h2)
    请求从h2进来
    // Tell the calling process that we've mounted the directory.
    r = channel.signal_peer(0, ZX_USER_SIGNAL_0)
    vn->Serve(this, fbl::move(channel), ZX_FS_RIGHT_ADMIN);
        vfs->ServeConnection(fbl::make_unique<Connection>(vfs, fbl::WrapRefPtr(this), fbl::move(channel), flags));
            connection->Serve()
                wait_.set_object(channel_.get());
                    wait_.object = object;
                wait_.Begin(vfs_->async());
                    async_begin_wait(async, &wait_)
                        async->ops->v1.begin_wait(async, wait)
                            zx_object_wait_async(
                                wait->object, loop->port, (uintptr_t)wait, wait->trigger, ZX_WAIT_ASYNC_ONCE);
                            把前面那个h2挂到async loop的port上
    h1交给Realm, 让Realm能够向component.Loader这个服务发送请求
    接下来仔细看看Realm的构造

component::Realm root_realm(nullptr, std::move(h1), kRootLabel)
    parent_是空
    构造一个缺省的default_namespace_
    Namespace(nullptr, this, nullptr)
        services_.set_backend(std::move(services_backend));   ServiceProviderBridge

        services_.AddService<Environment>(
                    [this](fidl::InterfaceRequest<Environment> request) {
                        environment_bindings_.AddBinding(this, std::move(request));
                    });

            AddServiceForName()
                name_to_service_connector_[service_name] = std::move(connector);

    构造RealmHub
    vfs获取缺省的async


    zx::channel::create(0, &svc_channel_server_, &svc_channel_client_)
        看看后面怎么用这个通道
    
    hub_.AddServices(default_namespace_->services());
    default_namespace_->services().set_backing_dir(std::move(host_directory))
        把通向"component.Loader"目录服务的通道赋值给ServiceProviderDirImpl的backing_dir_
      
    ServiceProviderPtr service_provider;
        ServiceProviderPtr是InterfacePtr<ServiceProvider>
    我们先看这个函数调用,
    service_provider.NewRequest()
        zx::channel::create(0, &h1, &h2)
            创建一个通道,一头给Bind(),另一头返回
        Bind(std::move(h1), async) 
            impl_->controller.reader().Bind(std::move(channel), async)
                impl里有ServiceProvider::Proxy_ 和  fidl::internal::ProxyController
                ProxyController里的MessageReader reader_监听h1
                wait_.object = channel_.get()
                async_begin_wait(async_, &wait_)

                controller收到消息会调用proxy_->Dispatch_()
        h1这头的处理目前看好像是fidl生成的缺省代码
        service_provider应该是作为客户端,所以还不需要处理返回的消息
        
        return InterfaceRequest<ServiceProvider>(std::move(h2));
            返回h2

    ServiceProviderBridge::AddBinding(h2)
        bindings_.AddBinding(this, std::move(request))
            this是ServiceProviderBridge自己 ImplPtr=ServiceProviderBridge*
            Binding(impl, request)
                Binding(std::forward<ImplPtr>(impl)
                    impl_(std::forward<ImplPtr>(impl)), stub_(&*this->impl())
                        stub_是ServiceProvider::Stub_,这个类的Dispatch_()会调用impl的ConnectToService(),
                        也就是ServiceProviderBridge::ConnectToService()
                controller_.reader().Bind(std::move(channel), async);
                    这里设置处理来自service_provider的请求
                    把channel挂到事件循环上
            
            bindings_.push_back()
                binding的类型是这个::fidl::Binding<ServiceProvider, InterfaceRequest<ServiceProvider>>

    建立了一个通道,一头让ServiceProvider::Proxy_处理,另一头给了ServiceProviderBridge添加binding,监听上

    service_provider.get()
        返回impl里的ServiceProvider::Proxy_     它继承了ServiceProvider
        
    下面这个ConnectToService的定义在connect.h里
    loader_ = ConnectToService<Loader>(ServiceProvider::Proxy_)    模板类型是Loader, interface_name="component.Loader"
        这里service_provider是ServiceProvider::Proxy_

        InterfacePtr<Loader> interface_ptr
        先看interface_ptr.NewRequest()
            创建通道,一头h1是Loader::Proxy_缺省处理,另一头h2返回

        service_provider->ConnectToService("component.Loader",h2)
            这个函数是fidl生成的
            controller_->Send
            把h2发送给了ServiceProviderDirImpl,因为之前通道的另一头给了ServiceProviderDirImpl::AddBinding(h2)
            ServiceProviderDirImpl::ConnectToService()会被调用,把h2发送给RootLoader::AddBinding监听起来
            fdio_service_connect_at()
                RootLoader::AddBinding

            返回InterfacePtr<Loader>
    上面这段代码的含义就是通过service_provider建立和Loader这个服务的联系
    service_provider是个客户端,它被AddBinding到ServiceProviderBridge里。
    ServiceProviderDirImpl本身是个服务,它的唯一的服务内容就是去ConnectToService()

PublishRootDir(&root_realm, &publish_vfs);
    request = zx_get_startup_handle(PA_DIRECTORY_REQUEST)
    dir(fbl::AdoptRef(new fs::PseudoDir()
    svc = fbl::AdoptRef(new fs::Service([root](zx::channel channel) {
        return root->BindSvc(std::move(channel));
    }));
        碰到Serve请求会调用这个函数。
        fdio_service_clone_to(root_realm->svc_channel_client_.get(),channel.release());
            zxrio_connect(svc, srv, ZXRIO_CLONE, ZX_FS_RIGHT_READABLE |ZX_FS_RIGHT_WRITABLE, 0755, "");
                把channel发送给svc
        这里有点奇怪,svc_channel_client_的另一头要等到sysmgr调用CreateNestedEnvironment才会被监听


    dir->AddEntry("hub", root->hub_dir());
        hub_.dir()
           dir_
    dir->AddEntry("svc", svc);

publish_vfs_.ServeDirectory(publish_dir_, zx::channel(args.pa_directory_request));
  把从devmgr发来的通道挂到目录服务上
  OpenVnode(flags, &vn)
    (*vnode)->Open(flags, &redirect)
      PseudoDir::Open()啥也不干
  vn->Serve(this, fbl::move(channel), ZX_FS_RIGHT_ADMIN)
    vfs->ServeConnection(fbl::make_unique<Connection>(vfs, fbl::WrapRefPtr(this), fbl::move(channel), flags));
      connection->Serve();
        wait_.set_object(channel_.get());
        wait_.Begin(vfs_->async());

run_sysmgr()
  fuchsia::sys::LaunchInfo launch_info;
    LaunchInfo是fidl生成的, launcher.fidl
  sysmgr.NewRequest()    InterfacePtr<ComponentController>
    Bind(std::move(h1), async)
      impl_->controller.reader().Bind(std::move(channel), async);
        wait_.object = channel_.get();
        async_begin_wait(async_, &wait_)
    ComponentController_Proxy缺省处理
    返回另一头h2

  root_realm.CreateComponent(std::move(launch_info),h2);
    scheme = "file"
    cb = lambda
    loader_->LoadComponent(url, cb)
        Loader_Proxy::LoadComponent()
            把cb发送给root loader

RootLoader::LoadComponent(fidl::StringPtr url,
                        LoadComponentCallback callback)
    CreateComponentWithProcess(
                    std::move(package), std::move(launch_info),
                    std::move(controller), std::move(ns));

        builder.AddServices(std::move(svc));
            PushDirectoryFromChannel("/svc", std::move(services));
        flat namespace没有
        channels = Util::BindDirectory(&launch_info);
          launch_info->directory_request = std::move(exported_dir_server);

        CreateProcess(job_for_child_, std::move(executable), url,
                    std::move(launch_info), zx::channel(), builder.Build());
          fdio_spawn_vmo(job.get(), flags, data.vmo().release(), argv.data(),
                    nullptr, actions.size(), actions.data(),
                    process.reset_and_get_address(), err_msg);

              zx_channel_create(0, &launcher, &launcher_request)  
              fdio_service_connect("/svc/fuchsia.process.Launcher", launcher_request);
                fdio_ns_connect(fdio_root_ns, svcpath,
                               ZX_FS_RIGHT_READABLE | ZX_FS_RIGHT_WRITABLE, h);
                  ns_walk_locked(&vn, &path)
                  fdio_open_at(vn->remote, path, flags, h); 
                    zxrio_connect(dir, h, ZXRIO_OPEN, flags, 0755, path);
                  建立和process launcher的通道
              send_string_array(launcher, fuchsia_process_LauncherAddArgsOrdinal, argv);
              send_handles(launcher, handle_capacity, flags, job, action_count, actions, err_msg);
              send_namespace(launcher, name_count, name_len, flat, action_count, actions, err_msg);
              zx_channel_call(launcher, 0, ZX_TIME_INFINITE, &args,
                                 &actual_bytes, &actual_handles);

fdio_service_connect("/svc/.", h1.release())
  fdio_ns_connect(fdio_root_ns, svcpath,
                               ZX_FS_RIGHT_READABLE | ZX_FS_RIGHT_WRITABLE, h);
    ns_walk_locked(&vn, &path)       
------------

收到从sysmgr来的rpc:
Namespace::CreateNestedEnvironment()
    realm_->CreateNestedJob(std::move(host_directory), std::move(environment),
                          std::move(controller), label);
        host_directory是sysmgr开过来的通道

        controller = EnvironmentControllerImpl(
                                        std::move(controller_request),
                                        std::make_unique<Realm>(this, std::move(host_directory), label));
        child = controller->realm();
            这是新建的child realm
        child->default_namespace_->services().ServeDirectory(
                std::move(root_realm->svc_channel_server_));
            让子realm来服务svc_channel_server_
            前面PublishRootDir把devmgr发来的通道发给了svc_channel_client_, 所以这里child realm会
            服务devmgr的请求





------------------------------------------------------


下面这些结构用来模拟unix fd io

typedef struct {
    mtx_t lock;
    mtx_t cwd_lock;
    bool init;
    mode_t umask;
    fdio_t* root;
    fdio_t* cwd;
    fdio_t* fdtab[FDIO_MAX_FD];
    fdio_ns_t* ns;
    char cwd_path[PATH_MAX];
} fdio_state_t;

extern fdio_state_t __fdio_global_state;

#define fdio_lock (__fdio_global_state.lock)
#define fdio_root_handle (__fdio_global_state.root)
#define fdio_cwd_handle (__fdio_global_state.cwd)
#define fdio_cwd_lock (__fdio_global_state.cwd_lock)
#define fdio_cwd_path (__fdio_global_state.cwd_path)
#define fdio_fdtab (__fdio_global_state.fdtab)
#define fdio_root_init (__fdio_global_state.init)
#define fdio_root_ns (__fdio_global_state.ns)

typedef struct async_dispatcher async_t;

static const async_ops_t async_loop_ops = {
    .version = ASYNC_OPS_V1,
    .reserved = 0,
    .v1 = {
        .now = async_loop_now,
        .begin_wait = async_loop_begin_wait,
        .cancel_wait = async_loop_cancel_wait,
        .post_task = async_loop_post_task,
        .cancel_task = async_loop_cancel_task,
        .queue_packet = async_loop_queue_packet,
        .set_guest_bell_trap = async_loop_set_guest_bell_trap,
    },
};

\end{verbatim}