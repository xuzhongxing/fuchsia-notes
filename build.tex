\section{Fuchsia Build}

fx set arm64

fx调用gn工具处理.gn文件,生成.ninja文件。ninja根据.ninja文件进行构建。

https://chromium.googlesource.com/chromium/src/tools/gn/

https://ninja-build.org/

\begin{verbatim}
/home/xzx/fuchsia/buildtools/gn gen /home/xzx/fuchsia/out/x64 
  --check '--args=target_cpu="x64" fuchsia_packages=["topaz/packages/default",]'

gn gen是生成ninja文件。
--args=设置一些变量供gn使用。
  target_cpu="x64" 
  fuchsia_packages=["topaz/packages/default",]

gn的缺省变量:
   - host_cpu
   - host_os
   - current_cpu
   - current_os
   - target_cpu
   - target_os

gn启动之后第一件事是寻找当前目录以及父目录里的.gn文件,确定源码树的根。从.gn文件开始执行。
.gn可能会定义secondary_source目录。gn也会去这里找文件。
寻找buildconfig文件,执行。
执行//BUILD.gn, fuchsia设置了root = "//build/gn", 所以会去这里找BUILD.gn
root  
  Label of the root build target. The GN build will start by loading the
  build file containing this target name. This defaults to "//:" which will
  cause the file //BUILD.gn to be loaded.

当一个target完整之后,输出ninja文件。
最后输出build.ninja


label包括target, config, toolchain

All targets encountered in the default toolchain (see "gn help toolchain")
will have build rules generated for them, even if no other targets reference
them.

buildconfig [required]
      Path to the build config file. This file will be used to set up the
      build file execution environment for each toolchain.

gn 定义变量:
var = value

target_toolchain = "//build/toolchain/fuchsia:x64"
\end{verbatim}