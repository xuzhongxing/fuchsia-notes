\section{Bootloader}

让make输出详细的编译过程。
\begin{verbatim}
make V=1 arm64 > make.log
\end{verbatim}

在qemu里运行Zircon的命令为:
\begin{verbatim}
qemu-system-aarch64 -m 2048 -nographic -net none -smp 4 \
    -kernel /Users/xzx/zircon/build-arm64/qemu-zircon.bin \
    -machine virtualization=true -cpu cortex-a53 \
    -machine virt,gic_version=3 \
    -initrd /Users/xzx/zircon/build-arm64/bootdata.bin \
    -append 'TERM=xterm-256color \
    kernel.entropy-mixin=a7324cb430cbac8... \
    kernel.halt-on-panic=true '
\end{verbatim}

\begin{verbatim}
./build-arm64/tools/mkbootfs --target=boot -c -o ./build-arm64/bootdata.bin 
  ./build-arm64/bootfs.manifest 
\end{verbatim}

可以看出内核文件是qemu-zircon.bin。我们在make.log里找一下这个文件是怎么来的。
\begin{verbatim}
./build-arm64/tools/mkbootfs -o build-arm64/qemu-zircon.bin \
    ./build-arm64/zircon.bin --header ./build-arm64/qemu-boot-shim.bin \
    --header-align 65536
\end{verbatim}

mkbootfs会把qemu-zircon.bin所代表的内核映像写在offset为65536的地方。在之前写入的是
boot-shim的内容。qemu-zircon.bin的布局实际上是由image.S决定的。

\subsection{mkbootfs一些细节}
\begin{enumerate}
  \item \verb!system = true, ramdisk = false, header\_align = 65536!
  \item 调用\verb!import_file(path, system, ramdisk)!
    \begin{enumerate}
      \item 接着调用\verb!import_file_as(fn, ITEM_BOOTDATA, hdr.length, &hdr)!
      \item 调用\verb!import_directory_entry("bootdata", fn, &s)!,构造一个fsentry, length是输入文件的长度。
      \item 构造\verb!first_item, type=ITEM_BOOTDATA, first, last都指向fsentry!
    \end{enumerate}
  \item 进入\verb!write_bootdata(output_file, first_item, header_path, header_align)!
    \begin{enumerate}
      \item 首先把header原封不动的写入目标文件
      \item 然后seek到header align的位置
      \item 然后再留出一个\verb!bootdata_t!的位置
      \item copybootdatafile(fd, item->first->srcpath, item->first->length)
        \begin{enumerate}
          \item 首先把\verb!BOOTDATA_CONTAINER!读到hdr里
          \item 然后把剩下的内核映像原封不动的写入目标文件
       \end{enumerate}
      \item 把\verb!BOOTDATA_CONTAINER!再写入之前留出来的\verb!bootdata_t!的位置
    \end{enumerate}
\end{enumerate}


qemu-boot-shim.bin的来历:
\begin{verbatim}
./prebuilt/downloads/gcc/bin/aarch64-elf-objcopy -O binary \
    build-arm64/boot-shim/qemu/boot-shim.elf build-arm64/qemu-boot-shim.bin
\end{verbatim}

boot-shim.elf的来历:
\begin{verbatim}
./prebuilt/downloads/gcc/bin/aarch64-elf-ld -nostdlib --build-id -z noexecstack \
    -z max-page-size=4096 --gc-sections  --emit-relocs --build-id=none \
    ./build-arm64/boot-shim/qemu/boot-shim.S.o \
    ./build-arm64/boot-shim/qemu/boot-shim.c.o \
    ./build-arm64/boot-shim/qemu/debug.c.o \
    ./build-arm64/boot-shim/qemu/devicetree.c.o \
    ./build-arm64/boot-shim/qemu/util.c.o  \
    -T kernel/target/arm64/boot-shim/boot-shim.ld \
    -o build-arm64/boot-shim/qemu/boot-shim.elf
\end{verbatim}

boot-shim.S中\_start的地址是0。所以qemu-zircon.bin加载之后,自然就从boot-shim.S的\_start
开始执行。那么,内核的\_start又是如何找到的呢?注意到有这样一段代码:

\begin{verbatim}
    // x0: pointer to device tree
    // x1: pointer to kernel bootdata container
    bl      boot_shim

    // kernel entry point is returned in x0
    mov     tmp, x0

    // pass bootdata to kernel in x0
    adr     x0, bootdata_return
    ldr     x0, [x0]

    br      tmp
\end{verbatim}

也就是说,boot\_shim函数应该会找到kernel的entry point。我们去看一下。
\begin{verbatim}
return kernel_base + kernel->data_kernel.entry64;
\end{verbatim}
这里kernel\_base就是KERNEL\_ALIGN=65536。
boot\_shim函数从bootdata中获得入口地址。bootdata是mkbootfs这个工具填写
到启动映像里的。在image.S的头上就是bootdata。


\begin{verbatim}
// BOOTDATA_KERNEL payload (bootdata_kernel_t)
DATA(_bootdata_kernel_payload)
    // The boot-shim code expects this to be an offset from the beginning
    // of the load image, whatever the kernel's virtual address.
    .quad IMAGE_ELF_ENTRY - _bootdata_file_header
    .quad 0
END_DATA(_bootdata_kernel_payload)
\end{verbatim}

从而,上面的br tmp能够正确跳转到内核入口。

\verb!boot-shim.S!文件里设置了stack pointer,指向了boot-shim之后,kernel之前的
一块大小为4k的内存区域。

zircon.bin的来历:
\begin{verbatim}
./prebuilt/downloads/gcc/bin/aarch64-elf-objcopy \
    -O binary build-arm64/zircon-image.elf build-arm64/zircon.bin
\end{verbatim}

zircon-image.elf的来历:
\begin{verbatim}
./prebuilt/downloads/gcc/bin/aarch64-elf-ld -nostdlib --build-id -z noexecstack \
    -z max-page-size=4096 --gc-sections --build-id=none \
	-o build-arm64/zircon-image.elf -T kernel/image.ld \
    --just-symbols ./build-arm64/zircon.elf \
	./build-arm64/kernel-vars.ld ./build-arm64/zircon.image.o
\end{verbatim}

链接脚本image.ld内容:
\begin{verbatim}
ENTRY(IMAGE_ELF_ENTRY)

SECTIONS {
    . = IMAGE_LOAD_START;

    .load_image : {
        KEEP(*(.text))
    } :load_image

    /*
     * When a boot loader is actually using the ELF headers, it needs to
     * know how much memory to reserve after the load image (p_filesz is
     * the load image, and p_memsz > p_filesz to indicate the extra space
     * to reserve).  This ensures that the segment has the right p_memsz.
     */
    .bss : {
        . += ABSOLUTE(IMAGE_MEMORY_END) - ABSOLUTE(.);
	ASSERT(ABSOLUTE(.) == ABSOLUTE(IMAGE_MEMORY_END), "image layout bad");
    }
}

PHDRS {
    load_image PT_LOAD FLAGS(7); /* PF_R|PF_W|PF_X */
}
\end{verbatim}

ENTRY指定了程序入口的虚拟地址,它对应于ELF Header中的entry项。
IMAGE\_ELF\_ENTRY的定义在kernel/arch/arm64/start.S中。
\begin{verbatim}
// This symbol is used by image.S.
.global IMAGE_ELF_ENTRY
IMAGE_ELF_ENTRY = _start
\end{verbatim}

实际上在最终生成的内核映像中ELF结构是不存在的,所以上面的ENTRY信息
并没有实际的用处。

IMAGE\_LOAD\_START是在kernel.ld中定义的符号,它的值是-4GB
\begin{verbatim}
. = KERNEL_BASE;
PROVIDE_HIDDEN(__code_start = .);

...

IMAGE_LOAD_START = __code_start;
\end{verbatim}

KERNEL\_BASE定义在kernel/arch/arm64/rules.mk

\begin{verbatim}
KERNEL_BASE := 0xffffffff00000000
\end{verbatim}

整个内核映像的起始虚拟地址是-4GB,但是入口并不是这个地址,而是start.S文件中
的\_start。我们在上面已经看到boot-shim是如何找到这个入口地址的。

这里zircon.elf只是用来获得symbol的地址,并不添加到zircon-image.elf里。真正的
内核映像文件是zircon.image.o

\begin{verbatim}
./prebuilt/downloads/gcc/bin/aarch64-elf-gcc   \
	-O2 -g -fdebug-prefix-map=/Users/xzx/zircon=. -finline \
    -include ./build-arm64/config-global.h -Wall -Wextra -Wno-multichar \
    -Werror -Wno-error=deprecated-declarations -Wno-unused-parameter \
    -Wno-unused-function -Werror=unused-label -Werror=return-type -fno-common \
    -ffunction-sections -fdata-sections -Wno-nonnull-compare -ffreestanding \
    -include ./build-arm64/config-kernel.h -Wformat=2 -Wvla -Wformat-signedness \
    -fno-exceptions -fno-unwind-tables -fno-omit-frame-pointer -mgeneral-regs-only \
    -fPIE -include kernel/include/hidden.h   -mcpu=cortex-a53 -ffixed-x18 \
    -Isystem/public -Isystem/private -I./build-arm64/gen/global/include \
    -I./build-arm64 -Ikernel/include -Ikernel/platform/generic-arm/include \
    -Ikernel/arch/arm64/include -Ikernel/top/include \
    -Ikernel/dev/hdcp/amlogic_s912/include \
    -Ikernel/dev/interrupt/arm_gic/common/include \
    -Ikernel/dev/interrupt/arm_gic/v2/include \
    -Ikernel/dev/interrupt/arm_gic/v2/include \
    -Ikernel/dev/interrupt/arm_gic/v3/include \
    -Ikernel/dev/iommu/dummy/include -Ikernel/dev/pcie/include \
    -Ikernel/dev/pdev/include -Ikernel/dev/pdev/power/include \
    -Ikernel/dev/power/hisi/include -Ikernel/dev/psci/include \
    -Ikernel/dev/timer/arm_generic/include -Ikernel/dev/uart/amlogic_s905/include \
    -Ikernel/dev/uart/nxp-imx/include -Ikernel/dev/uart/pl011/include \
    -Ikernel/kernel/include -Isystem/ulib/bitmap/include \
    -Ikernel/lib/bitmap/include -Ikernel/lib/cbuf/include \
    -Ikernel/lib/debugcommands/include -Ikernel/lib/debuglog/include \
    -Ikernel/lib/ktrace/include -Ikernel/lib/memory_limit/include \
    -Ikernel/lib/mtrace/include -Ikernel/lib/userboot/include \
    -Ikernel/lib/version/include -Ikernel/object/include \
    -Ikernel/platform/include -Ikernel/syscalls/include \
    -Ikernel/target/include -Ikernel/tests/include \
    -Ikernel/dev/interrupt/include -Ikernel/dev/pdev/interrupt/include \
    -Ikernel/dev/pdev/uart/include -Ikernel/dev/udisplay/include \
    -Ikernel/lib/console/include -Ikernel/lib/counters/include \
    -Ikernel/lib/crypto/include -Ikernel/lib/debug/include \
    -Isystem/ulib/explicit-memory/include -Ikernel/lib/explicit-memory/include \
    -Ikernel/lib/fbl/include -Isystem/ulib/fbl/include \
    -Ikernel/lib/fbl/include -Ikernel/lib/fixed_point/include \
    -Ikernel/lib/header_tests/include -Ikernel/lib/heap/include \
    -Ikernel/lib/heap/include -Ikernel/lib/hypervisor/include \
    -Ikernel/lib/libc/include -Ikernel/lib/oom/include -Ikernel/lib/pci/include \
    -Ikernel/lib/pci/include -Ikernel/lib/pow2_range_allocator/include \
    -Isystem/ulib/region-alloc/include -Ikernel/lib/region-alloc/include \
    -Ikernel/lib/unittest/include -Ikernel/lib/user_copy/include \
    -Ikernel/lib/vdso/include -Isystem/ulib/zxcpp/include \
    -Ikernel/lib/zxcpp/include -Ikernel/vm/include \
    -Ikernel/arch/arm64/hypervisor/include -Ikernel/dev/hw_rng/include \
    -Ikernel/lib/gfx/include -Ikernel/lib/gfxconsole/include \
    -Ikernel/lib/heap/cmpctmalloc/include -Ikernel/lib/io/include \
    -Isystem/ulib/pretty/include -Ikernel/lib/pretty/include \
    -Ithird_party/ulib/cryptolib/include -Ithird_party/lib/cryptolib/include \
    -Ithird_party/lib/jitterentropy/include -Ithird_party/lib/jitterentropy/include\
    -Ithird_party/ulib/qrcodegen/include -Ithird_party/lib/qrcodegen/include \
    -Ithird_party/ulib/uboringssl/include -Ithird_party/lib/uboringssl/include \
    -I./build-arm64 \
	-c kernel/arch/arm64/image.S -MD -MP -MT build-arm64/zircon.image.o \
    -MF build-arm64/zircon.image.d -o build-arm64/zircon.image.o
\end{verbatim}

image.S文件里嵌入了真正的内核zircon.elf.bin

\begin{verbatim}
./prebuilt/downloads/gcc/bin/aarch64-elf-objcopy \
-O binary build-arm64/zircon.elf build-arm64/zircon.elf.bin
\end{verbatim}

真正的kernel是zircon.elf
\begin{verbatim}
./prebuilt/downloads/gcc/bin/aarch64-elf-ld -nostdlib --build-id -z noexecstack \
    -z max-page-size=4096 --gc-sections  --emit-relocs -T kernel/kernel.ld \
    build-arm64/kernel-vars.ld \
    build-arm64/kernel/platform/generic-arm/generic-arm.mod.o \
    build-arm64/kernel/arch/arm64/arm64.mod.o build-arm64/kernel/top/top.mod.o \
    build-arm64/kernel/dev/hdcp/amlogic_s912/amlogic_s912.mod.o \
    build-arm64/kernel/dev/interrupt/arm_gic/common/common.mod.o \
    build-arm64/kernel/dev/interrupt/arm_gic/v2/v2.mod.o \
    build-arm64/kernel/dev/interrupt/arm_gic/v3/v3.mod.o \
    build-arm64/kernel/dev/iommu/dummy/dummy.mod.o \
    build-arm64/kernel/dev/pcie/pcie.mod.o \
    build-arm64/kernel/dev/pdev/pdev.mod.o \
    build-arm64/kernel/dev/pdev/power/power.mod.o \
    build-arm64/kernel/dev/power/hisi/hisi.mod.o \
    build-arm64/kernel/dev/psci/psci.mod.o \
    build-arm64/kernel/dev/timer/arm_generic/arm_generic.mod.o \
    build-arm64/kernel/dev/uart/amlogic_s905/amlogic_s905.mod.o \
    build-arm64/kernel/dev/uart/nxp-imx/nxp-imx.mod.o \
    build-arm64/kernel/dev/uart/pl011/pl011.mod.o \
    build-arm64/kernel/kernel/kernel.mod.o \
    build-arm64/kernel/lib/bitmap/bitmap.mod.o \
    build-arm64/kernel/lib/cbuf/cbuf.mod.o \
    build-arm64/kernel/lib/debugcommands/debugcommands.mod.o \
    build-arm64/kernel/lib/debuglog/debuglog.mod.o \
    build-arm64/kernel/lib/ktrace/ktrace.mod.o \
    build-arm64/kernel/lib/memory_limit/memory_limit.mod.o \
    build-arm64/kernel/lib/mtrace/mtrace.mod.o \
    build-arm64/kernel/lib/userboot/userboot.mod.o \
    build-arm64/kernel/lib/version/version.mod.o \
    build-arm64/kernel/object/object.mod.o \
    build-arm64/kernel/platform/platform.mod.o \
    build-arm64/kernel/syscalls/syscalls.mod.o \
    build-arm64/kernel/target/target.mod.o \
    build-arm64/kernel/tests/tests.mod.o \
    build-arm64/kernel/dev/interrupt/interrupt.mod.o \
    build-arm64/kernel/dev/pdev/interrupt/interrupt.mod.o \
    build-arm64/kernel/dev/pdev/uart/uart.mod.o \
    build-arm64/kernel/dev/udisplay/udisplay.mod.o \
    build-arm64/kernel/lib/console/console.mod.o \
    build-arm64/kernel/lib/counters/counters.mod.o \
    build-arm64/kernel/lib/crypto/crypto.mod.o \
    build-arm64/kernel/lib/debug/debug.mod.o \
    build-arm64/kernel/lib/explicit-memory/explicit-memory.mod.o \
    build-arm64/kernel/lib/fbl/fbl.mod.o \
    build-arm64/kernel/lib/fixed_point/fixed_point.mod.o \
    build-arm64/kernel/lib/header_tests/header_tests.mod.o \
    build-arm64/kernel/lib/heap/heap.mod.o \
    build-arm64/kernel/lib/hypervisor/hypervisor.mod.o \
    build-arm64/kernel/lib/libc/libc.mod.o \
    build-arm64/kernel/lib/oom/oom.mod.o \
    build-arm64/kernel/lib/pci/pci.mod.o \
    build-arm64/kernel/lib/pow2_range_allocator/pow2_range_allocator.mod.o \
    build-arm64/kernel/lib/region-alloc/region-alloc.mod.o \
    build-arm64/kernel/lib/unittest/unittest.mod.o \
    build-arm64/kernel/lib/user_copy/user_copy.mod.o \
    build-arm64/kernel/lib/vdso/vdso.mod.o \
    build-arm64/kernel/lib/zxcpp/zxcpp.mod.o \
    build-arm64/kernel/vm/vm.mod.o \
    build-arm64/kernel/arch/arm64/hypervisor/hypervisor.mod.o \
    build-arm64/kernel/dev/hw_rng/hw_rng.mod.o \
    build-arm64/kernel/lib/gfx/gfx.mod.o \
    build-arm64/kernel/lib/gfxconsole/gfxconsole.mod.o \
    build-arm64/kernel/lib/heap/cmpctmalloc/cmpctmalloc.mod.o \
    build-arm64/kernel/lib/io/io.mod.o \
    build-arm64/kernel/lib/pretty/pretty.mod.o \
    build-arm64/third_party/lib/cryptolib/cryptolib.mod.o \
    build-arm64/third_party/lib/jitterentropy/jitterentropy.mod.o \
    build-arm64/third_party/lib/qrcodegen/qrcodegen.mod.o \
    build-arm64/third_party/lib/uboringssl/uboringssl.mod.o \
    -o build-arm64/zircon.elf
\end{verbatim}

kernel在链接的时候,以-4G为基地址。也就是说,内核里所有的东西
的虚拟地址都在-4G地址之上。

