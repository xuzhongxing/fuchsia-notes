\section{ARM架构}

\verb!sp!指的是el0的stack pointer。其他的stack pointer需要指明el级别。在Zircon里,
el1使用的是\verb!sp_el1!,但是在\verb!arm64_elX_to_el1!里,\verb!sp_el1!
被设置为与\verb!sp!相同的值。

\verb!pc!寄存器是专用寄存器,不能再被显式指代。

frame pointer不是专用寄存器,是X29,但是A64 Procedure Call Standard把X29定义为专门的
frame pointer. X30是link register.

异常级别定义程序运行的特权级别。

安全状态下处理器能访问安全内存地址空间。

\verb!scr_el3!寄存器定义EL0和EL1的安全状态。

ELR存放异常返回地址。PLR存放函数调用返回地址。

处理异常之前,处理器状态(PSTATE)会保存在SPSR中。

\subsection{关于MMU}

EL1有2个页表基地址指针寄存器:\verb!ttbr0_el1!和\verb!ttbr1_el1!。使用
哪一个由虚拟地址的最高几位决定,如果都是0则使用\verb!ttbr0_el1!,如果都是1则
使用\verb!ttbr1_el1!。具体检查多少位由\verb!tcr_el1!的t0sz和t1sz决定。
在Zircon里t0sz=22, t1sz=16.

IPS设置为1TB.

\verb!ESR_ELn!寄存器存放了当前的异常具体是什么异常的信息。

\subsection{Exceptions}
\begin{verbatim}
sync_exception #(ARM64_EXCEPTION_FLAG_LOWER_EL), 1

.macro sync_exception, exception_flags, from_lower_el_64=0
    start_isr_func
    regsave_long
    mrs x9, esr_el1
.if \from_lower_el_64
    // If this is a syscall, x0-x7 contain args and x16 contains syscall num.
    // x10 contains elr_el1.
    lsr x11, x9, #26              // shift esr right 26 bits to get ec
    cmp x11, #0x15                // check for 64-bit syscall
    beq arm64_syscall_dispatcher  // and jump to syscall handler
.endif
    // Prepare the default sync_exception args
    mov x0, sp
    mov x1, ARM64_EXCEPTION_FLAG_LOWER_EL
    mov w2, w9
    bl  arm64_sync_exception
    b  arm64_exc_shared_restore_long
.endm
\end{verbatim}