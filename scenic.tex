\section{scenic}

\begin{verbatim}
service_root通向"/svc/"

CreateStaticServiceRootHandle()
    fdio_service_connect("/svc/.", h1.release())
      fdio_ns_connect(fdio_root_ns, svcpath,
                               ZX_FS_RIGHT_READABLE | ZX_FS_RIGHT_WRITABLE, h);

        __libc_extensions_init里面会把fdio的各种handle设置好。
        这样app就可以用fdio_ns_connect来获取channel了
        如果root namespace里面没有"/svc/",则这个进程不能访问这个名字

StartupContext::StartupContext(zx::channel service_root, zx::channel directory_request) {
  incoming_services_.Bind(std::move(service_root));
    directory_ = std::move(directory);
  
  outgoing_.Serve(std::move(directory_request));
    vfs_.ServeDirectory(root_dir_, std::move(dir_request))
  
  environment_.NewRequest()  // InterfacePtr<Environment>::NewRequest()
    zx::channel::create(0, &h1, &h2)
    Bind(std::move(h1), async)
      impl_->controller.reader().Bind(std::move(channel), async);  ????
    
    return InterfaceRequest<Interface>(std::move(h2))

    把通道的一头挂在本地事件循环里,Environment服务的客户端。


  incoming_services_.ConnectToService(environment_.NewRequest());


  incoming_services_.ConnectToService(launcher_.NewRequest());
}

App::App(fuchsia::sys::StartupContext* app_context, fit::closure quit_callback)
    : scenic_(std::make_unique<Scenic>(app_context, std::move(quit_callback)))
    app_context->outgoing().AddPublicService(scenic_bindings_.GetHandler(this));
    
    scenic_->RegisterSystem<scenic::gfx::GfxSystem>()     // SystemT = scenic::gfx::GfxSystem
        system = new GfxSystem(SystemContext(app_context_, quit_callback_.share())
            display_manager_.WaitForDefaultDisplay(cb1)
                display_watcher_.WaitForDisplay(cb2)
                    device_watcher_ = fsl::DeviceWatcher::Create("/dev/class/display-controller", cb3)
                        open("/dev/class/display-controller", O_DIRECTORY | O_RDONLY);
                          这里会触发display controller里面的DdkOpenAt()被调用,会给DisplayManager发送DisplaysChanged rpc
                        zx_channel_create(0, &wd.channel, &dir_watch_handle)
                        ioctl_vfs_watch_dir(dir_fd.get(), &wd)
                        device watcher会触发cb3的调用

如果/dev/class/display-controller存在,调用DeviceWatcher::Handler
DisplayWatcher::HandleDevice
  ioctl_display_controller_get_handle(fd.get(), dc_handle.reset_and_get_address())
    把ClientProxy的client handle拿到dc_handle里。
    调用cb2

      display_controller_.Bind(std::move(dc_handle));
      让wait开始处理display controller传来的消息:OnAsync()

由于刚才open display-controller的时候触发了display controller发送DisplaysChanged rpc
所以调用
DisplayManager::OnAsync()
  DisplayManager::DisplaysChanged()
    display_controller_->CreateLayer(&status, &layer_id_);
      rpc 调用display controller
      Client::HandleCreateLayer
        CreateLayer(&resp->layer_id);
          创建一个Layer
    display_controller_->SetDisplayLayers(display.id, std::move(fidl_layers))
      rpc:
      Client::HandleSetDisplayLayers
        把它放到对应的display的pending layers里
    default_display_ = std::make_unique<Display>
      创建新的Display    
    ClientOwnershipChange(owns_display_controller_);
      default_display_->ownership_event().signal(
          fuchsia::ui::scenic::displayOwnedSignal,
          fuchsia::ui::scenic::displayNotOwnedSignal);
        给这个event发信号。暂时不知道谁会处理。mark一下。

    display_available_cb_();  调用cb1
GfxSystem::Initialize()
  // 终于要跟Vulkan打交道了
  escher::VulkanIsSupported()
    首先检查是否有Vulkan支持
    CheckIfVulkanIsSupported();
      CreateVulkanInstance();
        vk::createInstance(&instance_info, nullptr, &instance)



\end{verbatim}